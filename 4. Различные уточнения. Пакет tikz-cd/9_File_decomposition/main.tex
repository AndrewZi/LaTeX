\documentclass[a4paper, 12pt]{article}

% LaTeX файл с преамбулой не должен содержать команду documentclass и \begin{document}. Эти команды должны находиться только в основном документе!

\usepackage[english, russian]{babel}
\usepackage[T2A]{fontenc}
\usepackage[utf8]{inputenc} % (.tex - можно не указывать)

% Справка:

% 1) \input{filename(.tex)}:
%   • подставляет текст непосредственно;
%   • может быть вложенным;
%   • полезен, чтобы не компилировать часть файлов,
%     т. к. можно убирать и комментировать строки с их вставкой в main.tex.

% 2) \include{filename(.tex)}:
%   • начинает текст с новой страницы;
%   • не может быть вложенным;
%   • чтобы не компилировать часть файлов можно добавить
%     \includeonly{filename_1(.tex), filename_1(.tex), ..., filename_n(.tex),}
%     (filename_1(.tex), filename_1(.tex), ..., filename_n(.tex) - обязательно должны быть указаны через \include{filename(.tex)} (!))
%     в файле преамбулы или перед \begin{document}


% При использовании вложенного обращения к файлам иерархия файловой системы
% ведёт отсчёт от корневой папки, содержащей main.tex (файл для компиляции документа)!

\begin{document}
    % Этот файл сам по себе не скомпилируется
% Не следует подключать файл преамбулы

\title{Интересная книга}
\author{Иванов И. И.}
% Будет отображена текущая дата
\maketitle
\pagebreak % При компиляции произойдёт просто вставка кода (.tex - можно не указывать)

    % Этот файл сам по себе не скомпилируется
% Не следует подключать файл преамбулы

Глава 1
\pagebreak % При компиляции произойдёт просто вставка кода (.tex - можно не указывать)

    % Этот файл сам по себе не скомпилируется
% Не следует подключать файл преамбулы

Глава 2 % При компиляции произойдёт просто вставка кода (.tex - можно не указывать)

    % Этот файл сам по себе не скомпилируется
% Не следует подключать файл преамбулы

Глава 3 % При компиляции произойдёт вставка кода с новой страницы (.tex - можно не указывать)

    % Этот файл сам по себе не скомпилируется
% Не следует подключать файл преамбулы

Глава 4 % При компиляции произойдёт вставка кода с новой страницы (.tex - можно не указывать)
\end{document}