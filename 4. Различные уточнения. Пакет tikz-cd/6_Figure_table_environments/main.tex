\documentclass[a4paper, 12pt]{article}

% Этот файл сам по себе не скомпилируется

\usepackage[T2A]{fontenc}
\usepackage[utf8]{inputenc}
\usepackage[english, russian]{babel} % Лучше, когда после fontenс и inputenc

\usepackage{tikz} % Для данного раздела везде

\usepackage{graphicx}

\usepackage{wrapfig} % Для wrapfigure и wraptable

% Нумерация у figure и wrapfigure; table и wraptable - общая (!)

% Справка по идентификаторам расположения, регистр важен (!):
% h - даёт рекомендацию LaTeX разместить объект как можно ближе к месту объявления, но в соответствии с его внутренними правилами;
% t - размещает объект вверху страницы;
% b - размещает объект внизу страницы;
% p - размещает объект на специальной странице для всех таких объектов;
% H - размещает объект прямо в месте объявления.

\begin{document}
    \begin{figure}[h] % <-- h - идентификатор расположения. См. справку выше
        \includegraphics[width = 0.5\textwidth]{image.png}
        \caption{Множество Мандельброта} % Подпись под изображением
        \label{fig:figure2} % Создание указателя на объект
    \end{figure}


    \begin{table}[p] % <-- p - идентификатор расположения. См. справку выше
        \begin{tabular}{|c|c|}
            a & b\\
            c & d
        \end{tabular}
        \caption{Какие-то данные} % Подпись под таблицей
        \label{tab:table1} % Создание указателя на объект
    \end{table}

    \begin{table}[p] % <-- p - идентификатор расположения. См. справку выше
        \begin{tabular}{|c|c|}
            e & f\\
            g & h
        \end{tabular}
        \caption{Какие-то другие данные} % Подпись под таблицей
        \label{tab:table2} % Создание указателя на объект
    \end{table}

    \setcounter{table}{7} % Изменение счётчика table (отвечает за нумерацию таблиц в документе)

    \begin{table}[p] % <-- p - идентификатор расположения. См. справку выше
        \begin{tabular}{|c|c|}
            i & g\\
            k & l
        \end{tabular}
        \caption{Ещё какие-то данные} % Подпись под таблицей
        \label{tab:table8} % Создание указателя на объект
    \end{table}

    \begin{table}[p] % <-- p - идентификатор расположения. См. справку выше
        \begin{tabular}{|c|c|}
            m & n\\
            o & p
        \end{tabular}
        \caption{Какие-то дополнительные данные} % Подпись под таблицей
        \label{tab:table9} % Создание указателя на объект
    \end{table}

    Некоторый текст, который должен оптекать рисунок\dots
    \begin{wrapfigure}{r}{100pt}
    % r - расположение объекта (l - слева строки, r - справа строки); 100pt - ширина объекта
        \includegraphics[width = 0.5\textwidth]{image.png}
        \caption{Множество Мандельброта} % Подпись под изображением
        \label{fig:figure3} % Создание указателя на объект
    \end{wrapfigure}

    % Некоторый текст, который должен оптекать таблицу\dots
    % \begin{wraptable}{l}{100pt}
    % l - расположение объекта (l - слева строки, r - справа строки); 100pt - ширина объекта
    %     \begin{tabular}{|c|c|}
    %         x & y\\
    %         z & t
    %     \end{tabular}
    %     \caption{Какие-то переменные} Подпись под таблицей
    %     \label{tab:table0} Создание указателя на объект
    % \end{wraptable}
\end{document}