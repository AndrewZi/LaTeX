\documentclass[a4paper, 12pt]{article}

% Этот файл сам по себе не скомпилируется

\usepackage[T2A]{fontenc}
\usepackage[utf8]{inputenc}
\usepackage[english, russian]{babel} % Лучше, когда после fontenс и inputenc

\usepackage{tikz} % Для данного раздела везде

\usepackage{hyperref} % Необходимый пакет

\hypersetup{ % Настройка ссылок (размещается или в файле преамбулы или в main.tex до \begin{document})
    unicode = true, % Юникод в названиях разделов в PDF (в частности, для русского языка)
    colorlinks = true, % Цветные ссылки вместо ссылок в рамках
    linkcolor = blue, % Цвет внутренних ссылок
    citecolor = green, % Цвет ссылок на библиографию
    filecolor = magenta, % Цвет ссылок на файлы
    urlcolor = blue % Цвет ссылок на URL
    % ... <-- Есть настройки цветов ссылок других типов
}

% Задание других цветов:

% 1) Встроенные возможности:
% black!70!blue <-- 70% - чёрный, 30% - белый

% 2) Используя пакет xcolor:
% \definecolor{darkblue}{rgb}{0.6, 0.6, 0.9} <-- darkblue - новый цвет (можно использовать в коде)

\begin{document}
    Всё в комментариях\dots
\end{document}