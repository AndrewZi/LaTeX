\documentclass[a4paper, 12pt]{article}

% LaTeX файл с преамбулой не должен содержать команду documentclass и \begin{document}. Эти команды должны находиться только в основном документе!

\usepackage[english, russian]{babel}
\usepackage[T2A]{fontenc}
\usepackage[utf8]{inputenc}

\usepackage{graphics} % Для \scalebox

% Справка по "боксам" (у некоторых контейнеров есть опциональные аргументы [], они опущены):
% \parbox{width}{text} - невидимый контейнер (фиксирует содержимое по ширине);
% \framebox{text} - контейнер в рамке (заворачивает содержимое в рамку);
% \colorbox{color}{text} - цветной контейнер (подкрашивает фон контейнера);
% \raisebox{lift}{text} - меняет вертикальное положение содержимого (поднимает или опускает содержимое на значение lift (оно может быть как положительным (тогда сожержимое поднимается вверх), так и отрицательным (тогда сожержимое опускается вниз) в любых допустимых в LaTeX единицах измерения));
% \rotatebox{degress}{text} - вращает содержимое (вращает содержимое на значение degress (оно может быть как положительным (тогда сожержимое поворачивается по часовой стрелке), так и отрицательным (тогда сожержимое поворачивается против часовой стрелки));
% \scalebox{scale}{text} - масштабирует содержимое;
% ... и др.

\newcommand{\Chi}{
    \raisebox{0.5ex} % Подъём \scalebox{1.1}{$\chi$} на 0.5 высоты "x"
    {\scalebox{1.1}{$\chi$} % Увеличение $\chi$ в 1.1 раза
    }
}

% vvv \hbox, \vbox - старые контейнеры из TeX
\newcommand{\divby}{ % Создание символа делимости
    \mathrel{\vbox{ % \mathrel задаёт специальные пробелы слева и справа от создаваемого символа (в формате математического отношения)
        \baselineskip=0.65ex % установка расстояния между базовыми линиями в \vbox (0.65 высоты "x")
        \lineskiplimit=0pt % установка предела для вертикального расстояния между строками текста в \vbox. Если расстояние между строками меньше этого значения, то LaTeX не будет добавлять дополнительное пространство.
        % Установка ^^^ этого значения в 0pt позволяет избежать дополнительных пробелов между точками.
        \hbox{.}\hbox{.}\hbox{.} % содержимое \vbox
    }}
}
% Или (много проще)

\newcommand{\anodivby}{
    \mathrel{ % задаёт специальные пробелы слева и справа от создаваемого символа (в формате математического отношения)
        \scalebox{0.85}{$\vdots$} % Увеличение $\vdots$ в 0.8 раза (фактически уменьшение)
    }
}

\begin{document}
    $\chi$abc % <-- по умолчанию \chi ниже уровня написания abc

    $\Chi$abc % <-- Теперь лучше


    \[4 \divby 2\]
    % Или

    \[4 \anodivby 2\]
\end{document}