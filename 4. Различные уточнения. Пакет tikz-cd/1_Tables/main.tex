\documentclass[a4paper, 12pt]{article}

% LaTeX файл с преамбулой не должен содержать команду documentclass и \begin{document}. Эти команды должны находиться только в основном документе!

\usepackage[english, russian]{babel}
\usepackage[T2A]{fontenc}
\usepackage[utf8]{inputenc}

\usepackage{array} % Для {p{2cm}|m{3cm}|b{4cm}}

\usepackage{tabularx} % Для использования X как ширины колонок таблиц

% tabbing - устаревший способ создания таблиц, не использовать!

\begin{document}
    % p - выравнивание по верхнему краю
    % m - выравнивание по середине
    % b - выравнивание по нижнему краю
    % cm - сантиметры
    % По умолчанию выранивание vvv по ширине

    \begin{tabular}{|p{2cm}|m{3cm}|b{4cm}|}
        text & text & text
    \end{tabular}\\

    \begin{tabular}{|p{2cm}|m{3cm}|c|} % Для столбца с c подберёт оптимальную ширину
        text & text & text
    \end{tabular}\\

    % >{\bfseries \itshape Exp.} - код будет применён к каждому элементу столбца до (>{}) текста столбца
    \begin{tabular}{|p{2cm}|m{3cm}|>{\bfseries \itshape Exp. }c|} % Для столбца с "c" подберётся оптимальная ширина
        text & text & text\\
        text & text & text
    \end{tabular}\\

    % <{ End.} - код будет применён к каждому элементу столбца после (<{}) текста столбца
    \begin{tabular}{|p{2cm}|m{3cm}|c<{ End.}|} % Для столбца с "c" подберётся оптимальная ширина
        text & text & text\\
        text & text & text
    \end{tabular}\\

    % >{\bfseries \itshape Exp.} - код будет применён к каждому элементу столбца до (>{}) текста столбца
    % <{ End.} - код будет применён к каждому элементу столбца после (<{}) текста столбца
    \begin{tabular}{|p{2cm}|m{3cm}|>{\bfseries \itshape Exp. }c<{ End.}|} % Для столбца с "c" подберётся оптимальная ширина
        text & text & text\\
        text & text & text
    \end{tabular}\\

    % X - будет равная ширина колонок, вычесленная автоматически
    % \textwidth - ширина всей таблицы (в данном случае совпадает с шириной строки документа)
    \begin{tabularx}{\textwidth}{|X|X|X|}
        text text text & text text text & text text text
    \end{tabularx}\\
\end{document}