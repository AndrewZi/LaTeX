\documentclass[a4paper, 12pt]{article}

% LaTeX файл с преамбулой не должен содержать команду documentclass и \begin{document}. Эти команды должны находиться только в основном документе!

\usepackage[english, russian]{babel}
\usepackage[T2A]{fontenc}
\usepackage[utf8]{inputenc}

\usepackage{amsmath}

\usepackage{tikz-cd} % Для работы с коммутативными диаграммами


% Справка:

% \arrow{rr}{\varphi} - стрелка (специфично для tikz-cd):
% rr - на 2 ячейки вправо (r);
% \varphi - текст под стрелкой.

% \arrow[swap]{rr}{\varphi} - стрелка (специфично для tikz-cd):
% swap - инверсия отображения подписи стрелки;
% dr - на 1 ячейку вниз (d), на 1 ячейку вправо (r);
% \pi - текст над стрелкой.

% \arrow[dashrightarrow, swap]{dl}{\psi} - стрелка (специфично для tikz-cd):
% dashrightarrow - пунктирная стрелка;
% swap - инверсия отображения подписи стрелки;
% dl - на 1 ячейку вниз (d), на 1 ячейку влево (l);
% \psi - текст над стрелкой.

% \arrow[dashrightarrow, swap]{ur}{\theta} - стрелка (специфично для tikz-cd):
% dashrightarrow - пунктирная стрелка;
% swap - инверсия отображения подписи стрелки;
% ur - на 1 ячейку ввех (u), на 1 ячейку вправо (r);
% \theta - текст над стрелкой.


\begin{document}
    \[
        \begin{tikzcd}[column sep = large, row sep = huge] % <-- Между столбцами и строками запрашивается большое расстояние
            G \arrow{rr}{\varphi} \arrow[swap]{dr}{\pi} & & \mathrm{Im\varphi} \arrow[dashrightarrow, swap]{dl}{\psi}
            \\ % <-- \\ - разделитель строк, & - разделитель элементов строки
            & G / K \arrow[dashrightarrow, swap]{ur}{\theta} & % Стрелки от G/K к ImPhi и от ImPhi к G/K совпали...
        \end{tikzcd}
    \]
\end{document}