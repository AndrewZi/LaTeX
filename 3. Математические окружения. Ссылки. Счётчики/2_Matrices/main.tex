\documentclass[a4paper, 12pt]{article}

% Этот файл сам по себе не скомпилируется

\usepackage[T2A]{fontenc}
\usepackage[utf8]{inputenc}
\usepackage[english, russian]{babel} % Лучше, когда после fontenс и inputenc

\usepackage{tikz} % Для данного раздела везде

\usepackage{amsmath}

\usepackage{nicematrix}

\begin{document}
    \begin{align*}
        \begin{pmatrix} % (...) <-- матрица
            a & b\\
            c & d
        \end{pmatrix}
    \end{align*}

    \begin{align*}
        \begin{vmatrix} % |...| <-- определитель
            a & b\\
            c & d
        \end{vmatrix}
    \end{align*}

    \begin{align*}
        \left|\begin{vmatrix} % ||...||
            a & b\\
            c & d
        \end{vmatrix}\right|
    \end{align*}

    \[ % Отсутп между выражениями меньше, чем при align*
        \begin{pmatrix} % Большая матрица
            a_{11} & a_{12} & \cdots & a_{1n}\\ % \cdots - многоточие по центру строки
            a_{21} & a_{22} & \cdots & a_{2n}\\
            \vdots & \vdots & \ddots & \vdots\\
            % \vdots - вертикальное многоточие, \ddots - диагональное многоточие
            a_{n1} & a_{n2} & \cdots & a_{nn}\\
        \end{pmatrix}
    \]

    \begin{align*}
        \begin{pNiceArray}{cc|c} % Расширенная матрица
            a_{11} & a_{12} & b_1\\
            a_{21} & a_{22} & b_2\\
        \end{pNiceArray}
    \end{align*}
\end{document}