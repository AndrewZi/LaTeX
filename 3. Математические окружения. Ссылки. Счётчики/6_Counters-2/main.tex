\documentclass[a4paper, 12pt]{article}

% Этот файл сам по себе не скомпилируется

\usepackage[T2A]{fontenc}
\usepackage[utf8]{inputenc}
\usepackage[english, russian]{babel} % Лучше, когда после fontenс и inputenc

\usepackage{tikz} % Для данного раздела везде

\usepackage{enumitem} % Для [label=\Alph*] или [label=\Roman*] в enumerate

\renewcommand{\labelitemi}{\$} % Переопределение маркеров списков первого уровня вложенности

\renewcommand{\labelitemii}{$\int$} % Переопределение маркеров списков второго уровня вложенности

\begin{document}
    С латинскими буквами:

    \begin{enumerate} [label=\Alph*] % \Alph - заглавные буквы латинского алфавита
        \item item A
        \setcounter{enumi}{5}
        \item another item % Здесь счётчик enumi = 6
    \end{enumerate}
    

    С римскими числами:

    \begin{enumerate} [label=\Roman*] % \Roman - римские числа
        \item item A
        \setcounter{enumi}{5}
        \item another item % Здесь счётчик enumi = 6
    \end{enumerate}



    Ненумерованный список:
    % Можно разместить переопределения здесь:
    % т. е. перед фрагментом разметки, где эти переопределения должны быть применены
    % \renewcommand{\labelitemi}{\$} Переопределение маркеров списков первого уровня вложенности
    % \renewcommand{\labelitemii}{$\int$} Переопределение маркеров списков второго уровня вложенности
    
    \begin{itemize}
        \item \begin{itemize}
            \item Первый подэлемент % По умолчанию маркеруется тире
            \item Не первый подэлемент % По умолчанию маркеруется тире
        \end{itemize}
        \item Не первый элемент % По умолчанию маркеруется жирными точками
    \end{itemize}
\end{document}