\documentclass[a4paper]{article}

% LaTeX файл с преамбулой не должен содержать команду documentclass и \begin{document}. Эти команды должны находиться только в основном документе!

\usepackage[english, russian]{babel}
\usepackage[T2A]{fontenc}
\usepackage[utf8]{inputenc}

\usepackage{amsmath}

\begin{document}
    \begin{multline}
        \int \dots = \dots = \dots = \dots = \dots =\\
        \hspace{200pt} % В этом случае аргумент может быть, как положительный, так и отрицательный
        = \dots = \dots = \dots = \dots =\\
        % \hspace{-200pt} Здесь работать не будет!
        = \dots = \dots = \dots = 1
    \end{multline}

    \[
        \begin{split} % Аналог align
            \int \dots = \dots = \dots = \dots = \dots =\\
            = \dots = \dots = \dots = \dots =\\
            = \dots = \dots = \dots = 1
        \end{split}
    \]

    \begin{align*}
        F = \frac{dp}{dt} \tag{II ЗН} % \tag позволяет справа помечать математические выражения
    \end{align*}

    \[
        \begin{cases} % Использование cases не по назначению!
            2 + 2 = 4
        \end{cases}
    \]

    \[
        \begin{aligned} % Пример использования aligned
            2 + 2 = 4\\
            3 + 2 = 5
        \end{aligned}
    \]

    \begin{align*}
        \begin{cases} % Система уравнений
            ax + by = e,\\
            cx + dy = f.
        \end{cases}
    \end{align*}

    \[
        \left[\begin{aligned}
            &x + y = 5\\ % Выравнивание по &
            &2x + 5y = 36 % Выравнивание по &
        \end{aligned}\right. % \right. <-- . - нет правой квадратной скобки
    \]
\end{document}