\documentclass[a4paper, 12pt]{article}

% Этот файл сам по себе не скомпилируется

\usepackage[T2A]{fontenc}
\usepackage[utf8]{inputenc}
\usepackage[english, russian]{babel} % Лучше, когда после fontenс и inputenc

\usepackage{tikz} % Для данного раздела везде

\usetikzlibrary{calc} % Для автоматического вычисления положения объектов

\usepackage{tkz-euclide} % Продвинутая работа с геометрией (\tkzMarkAngle, \tkzMarkRightAngle, \tkzLabelArc, \tkzMarkSegment)

% Справка. Возможности calc:

% 1) Сложение и вычитание координат:
%   • \coordinate (A) at (1, 2);
%   • \coordinate (B) at (A + (2, 3)); % B будет в (3, 5)
%   • \coordinate (C) at (A - (1, 1)); % C будет в (0, 1)

% 2) Умножение и деление на скаляр:
%   • \coordinate (D) at (A * 2); % D будет в (2, 4)
%   • \coordinate (E) at (A / 2); % E будет в (0.5, 1)

% 3) Применение углов и расстояний:
%   • \coordinate (F) at ($(A) + (45:2cm)$); % F будет на 2 см под углом 45 градусов от A

% 4) Средние точки:
%   • \coordinate (G) at ($(A)!0.5!(B)$); % G будет в середине между A и B

% 5) Проекции и перпендикуляры:
%   • \coordinate (H) at ($(A)!(B)!(C)$); % H будет проекцией A на прямую BC

% 6) Использование относительных координат:
%   • \coordinate (I) at ($(A) + (1, 0)$); % I будет на 1 единицу вправо от A

% ... См. библиотека calc пакета tikz в Интернете

\begin{document}
    \begin{tikzpicture}
        \coordinate (A) at (0, 0);
        \coordinate (B) at (60:1.5); % Задание координаты точки B в полярных координатах
        \coordinate (C) at (3, 2);
        \coordinate (D) at ($(A) + (2, -1)$); % Задание координаты точки D с использованием относительных координат
        \draw [dashed] (A) -- (A |- B) -- (B); % (A |- B) <-- Точка на пересечении перпендикуляров к точкам A и B
        \draw [dotted] (B) -- (D) -- (C);
        \coordinate (F) at ($(D)!(B)!(C)$); % Получение точки на DC перпендикуляра к DC из точки B
        \draw (B) -- (F);
        \draw (A) -- ($(B)!0.5!(D)$); % Получение медианы к BD из точки A

        \tkzMarkAngle[mark = , arc = ll, size = 0.5](C,D,B) % Без ";" в конце (угловая дуга)
        % Здесь: mark = , <=> mark = none,
        % arc = ll <-- две дугообразные "душки" (arc = l <-- одна дугообразная "душка")
        % Обязательно:
        % Все точки, используемые в \tkzMarkAngle, \tkzMarkRightAngle, \tkzLabelArc, \tkzMarkSegment, ... должны быть инициализированы до вызова этих команд
        % \tkzMarkAngle[...](...) <-- без пробелов
        % (C,D,B) <-- без пробелов между C,; D, и B
        % size = 0.5 (нельзя 0.5pt, 0.5mm, 0.5cm и т. п.)

        \tkzMarkRightAngle(D,F,B) % Без ";" в конце (прямой угол, но для использования угол не обязательно должен быть прямым)
        \tkzLabelArc[pos = 0.4, above right = 0.3](C,D,B){\ \ \ \ \ $\alpha$} % Без ";" в конце (обозначение угла CDB за \alpha)
        % Расположение приходится подгонять всеми возможными способами... ("\ " - экранирование пробела)
        % vvv
        \tkzMarkSegment[mark = |](F,B) % Без ";" в конце (риска на отрезке, можно: mark = ||, mark = s)
        \tkzMarkSegment[mark = |](F,D) % Без ";" в конце (риска на отрезке, можно: mark = ||, mark = s)
        % ^^^ Таким образом на рисунке явно показали равенство сторон
    \end{tikzpicture}
\end{document}