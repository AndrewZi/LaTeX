\documentclass[a4paper, 12pt]{article}

% LaTeX файл с преамбулой не должен содержать команду documentclass и \begin{document}. Эти команды должны находиться только в основном документе!

\usepackage[english, russian]{babel}
\usepackage[T2A]{fontenc}
\usepackage[utf8]{inputenc}

\usetikzlibrary{through} % Для использования through

\begin{document}
    \tikz \coordinate (A) at (0, 0); % Создание точки A с координатами (0, 0) (в A записываются координаты, но ничего не будет отрисовано)
    \tikz \draw (A) -- (1, 0); % Отображается линия от точки A с координатами (0, 0) до (1, 0)

    \begin{tikzpicture} % Координаты (0, 0) теперь отсчитываются относительно окружения tikzpicture,
% поэтому начальная точка линии, отображённой ранее, не будет совпадать с точкой на окружности, построенной далее.
        % \coordinate (A) at (1, 1); (в таком случае A была бы переопределена)
        \coordinate (B) at (0.1, 0.3);
        \draw [fill] (A) circle (0.5pt); % Для наглядности точки A
        \draw [fill] (B) circle (0.5pt); % Для наглядности точки A
        \node[draw, circle through = (A)] at (B) {};
        % \node - узел;
        % draw - узел должен быть отрисован;
        % circle through = (A) <-- узел должен быть окружностью, проходящей через координату точку A;
        % at {B} - указывает позицию, в которой будет размещён узел;
        % {} - текст или иные элементы в узле (здесь они пустые, что означает, что узел не будет содержать текста или других элементов).
    \end{tikzpicture}
\end{document}