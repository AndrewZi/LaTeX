\documentclass[a4paper, 12pt]{article}

% LaTeX файл с преамбулой не должен содержать команду documentclass и \begin{document}. Эти команды должны находиться только в основном документе!

\usepackage[english, russian]{babel}
\usepackage[T2A]{fontenc}
\usepackage[utf8]{inputenc}

\usetikzlibrary{through, intersections} % Для through и intersections(!)

\begin{document}
    \begin{tikzpicture}
        \coordinate (A) at (0, 0);
        \coordinate (B) at (3, 0);
        \draw [name path = A--B] (A) -- (B); % Сразу создаются два узла и сразу соединяются
        % Имя пути: A--B (обязательно должны быть использованы, иначе произойдёт ошибка компиляции)
        
        \node (D) [name path = D, draw, circle through = (B),
            label = left:$D$] at (A) {};
        % Имя пути, созданного узлом D: D (обязательно должно быть использовано, иначе произойдёт ошибка компиляции)

        \node (E) [name path = E, draw, circle through = (A),
            label = right:$E$] at (B) {};
        % Имя пути, созданного узлом E: E (обязательно должно быть использовано, иначе произойдёт ошибка компиляции)

        \path [name intersections = {of = D and E, % в параметре of передаются объекты, пересечения которых должно быть инициализированы
        % (при этом они не обязательно должны быть отображены в документе)
            by = {[label = above:$C$]C, % Первое пересечение (в [...] задаются параметры отображения), инициализируется точка C
                [label = below:$C'$]C'}}]; % Второе пересение (в [...] задаются параметры отображения), инициализируется точка C'
        
        \draw [name path = C--C', color = red] (C) -- (C'); % color = red <-- Цвет отображаемого пути
        \path [name intersections = {of = A--B and C--C', by = F}]; % Так как нет [label = above:$F$] перед F в by, точка F на рисунке подписана не будет
    \end{tikzpicture}
\end{document}