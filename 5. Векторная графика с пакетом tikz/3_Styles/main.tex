\documentclass[a4paper, 12pt]{article}

% LaTeX файл с преамбулой не должен содержать команду documentclass и \begin{document}. Эти команды должны находиться только в основном документе!

\usepackage[english, russian]{babel}
\usepackage[T2A]{fontenc}
\usepackage[utf8]{inputenc}

\tikzset{help lines/.style = {very thin, color = #1!50}, help lines/.default = {black}}
% 1) color = #1!50 — это определение цвета.
% Здесь #1 — это специальная переменная, которая будет заменена на значение цвета при использовании стиля.
% !50 означает, что цвет будет полупрозрачным (50% от оригинального цвета).
% 2) help lines/.default = {black} <--  Эта строка задает значение по умолчанию для стиля help lines, если при его использовании не указано другое значение.
% В данном случае, если не укажете цвет при использовании стиля, он будет чёрным.

% Воздействие стиля на другой стиль vvv (help lines, "= black" можно было не писать, так как это цвет по умолчанию)
\tikzset{help grid/.style = {step = #1, help lines = black}, help grid/.default = {0.5cm}}
% 1) step = #1 — это параметр, который можно указать при использовании этого стиля.
% Он также будет заменён на конкретное значение.
% 2) help lines = black <-- Это означает, что стиль help grid будет использовать стиль help lines, но с чёрным цветом.
% 3) help grid/.default = {0.5cm} <-- Это значение по умолчанию для параметра step.
% Если не укажете значение при использовании стиля help grid, оно будет равно 0.5 см.

\begin{document}
    \tikz \draw [help grid] (-1.4, -1.4) grid (1.4, 1.4);

    \tikz \draw [help grid = 1cm] (-1.4, -1.4) grid (1.4, 1.4); % help grid = 2cm <-- задание параметра (#1)
\end{document}