\documentclass[a4paper, 12pt]{article}

% Этот файл сам по себе не скомпилируется

\usepackage[T2A]{fontenc}
\usepackage[utf8]{inputenc}
\usepackage[english, russian]{babel} % Лучше, когда после fontenс и inputenc

\usepackage{tikz} % Для данного раздела везде

\usetikzlibrary{calc}

\begin{document}
    \begin{tikzpicture}
        \coordinate [label = left:$A$] (A) at (0, 0);
        % Будет инициализирована точка A, а также отображена подпись "A" этой точки в документе

        \coordinate [label = right:$B$] (B) at (5, 4);
        % Будет инициализирована точка B, а также отображена подпись "B" этой точки в документе

        \draw (A) -- (B);

        \node [fill = red, inner sep = 2pt, label = above:$D$] % inner sep - внутренний отступ (внутреннее расстояние) между содержимым узла и его границей
        % Благодаря inner sep получаем небольшой квадрат
            (D) at ($(A) ! .5 ! (B) ! {sin(60) * 2} ! 90:(B)$) {}; % .5 <=> 0.5

        \draw (A) -- (D) -- (B);

        \node [fill = gray, inner sep = 2pt, label = right:$X$]
        % Благодаря inner sep получаем небольшой квадрат
            (X) at ($(A) ! .5 ! (B)$) {};  % .5 <=> 0.5

        \draw [very thick, dashed] (A) -- (X);
        \draw [very thick, dashed] (X) -- ($(X) ! {sin(60) * 2} ! 90:(B)$); % Можно было: \draw [very thick, dashed] (X) -- (D);
        % ^^^ $(X) ! {sin(60) * 2} ! 90:(B)$ <-- сразу вернёт точку (узел)
    \end{tikzpicture}
\end{document}