\documentclass[a4paper, 12pt]{article}

% Этот файл сам по себе не скомпилируется

\usepackage[T2A]{fontenc}
\usepackage[utf8]{inputenc}
\usepackage[english, russian]{babel} % Лучше, когда после fontenс и inputenc

\usepackage{tikz} % Для данного раздела везде

\tikzset{help lines/.style = {very thin, gray}} % Задание стиля вспомогательных линий (; не нужна (!))

\begin{document}
    \begin{tikzpicture}
        \draw (0, 0) circle (10pt); % Круг
        \draw (0, 0) ellipse (20pt and 10pt); % Эллипс
        \draw [rotate = 30] (0, 0) ellipse (20pt and 10pt); % Повёрнутый эллипс

        \draw [step = .5cm, gray, very thin] (-1.4, -1.4) grid (1.4, 1.4);
        % step = .5cm <=> step = 0.5cm (с нулём понятнее запись);
        % gray - цвет линий сетки (серый);
        % very thin - толщина линий сетки (очень тонкие).

        % Или
        % \tikzset{help lines/.style = {very thin, gray}} <-- (можно здесь, но лучше в файле преамбулы или перед \begin{document})
        % ^^^ Задание стиля вспомогательных линий (; не нужна (!))
        \draw [step = .5cm, help lines] (-1.4, -1.4) grid (1.4, 1.4);
    \end{tikzpicture}
\end{document}