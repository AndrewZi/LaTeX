\documentclass[a4paper, 12pt]{article}

% LaTeX файл с преамбулой не должен содержать команду documentclass и \begin{document}. Эти команды должны находиться только в основном документе!

\usepackage[english, russian]{babel}
\usepackage[T2A]{fontenc}
\usepackage[utf8]{inputenc}

% Справка. Виды стрелок:
% ->
% <-
% -latex
% <<-
% ->>
% см. стрелки tikz в Интернете

\begin{document}
    \tikz \draw [->] (0, 0) -- (2, 0); % --------> (сплошная линия, но стрелка некрасивая)
    \begin{tikzpicture}[scale = 2, > = stealth] % stealth - визуально приятная стрелка
        \draw [->] (0, 0) arc (180:30:10pt and 20pt); % Из-за stealth такая же стрелка "->" отрисуется по-другому
        % arc (180:30:10pt and 20pt):
        % arc - дуга;
        % 180:30: - Здесь указываются углы в градусах, которые определяют начало и конец дуги.
        % 10pt and 20pt <-- Первый радиус (10pt) - это радиус по горизонтали (по оси X), а второй радиус (20pt) - это радиус по вертикали (по оси Y).
        % Таким образом, получаем участок эллиптической дуги со стрелкой.
        \draw [-latex] (1, 0) -- + (1, 1); % Также визульно приятная стрелка в виде треугольника (не требует stealth).
        \draw [<<-, very thick] (3, 0) -- (3.5cm, 10pt) -- (4cm, 0pt)
        -- (4.5cm, 10pt);
        % (3.5cm, 10pt) <-- координата по Ox в физических единицах
        % (Если написать (3.5cm, 10) вместо (3.5cm, 10pt), отображение будет другим)
    \end{tikzpicture}
\end{document}