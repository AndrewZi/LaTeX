\documentclass[a4paper, 12pt]{article}

% LaTeX файл с преамбулой не должен содержать команду documentclass и \begin{document}. Эти команды должны находиться только в основном документе!

\usepackage[english, russian]{babel}
\usepackage[T2A]{fontenc}
\usepackage[utf8]{inputenc}

\begin{document}
% Библиотека tikz рассматривает все обращения к ней на своём (!) языке
    \tikz \draw (0pt, 0pt) -- (1in, 8pt); % ; - обязательна!
    \tikz \fill[orange] (1ex, 1ex) circle (1ex); % ; - обязательна!

    \begin{tikzpicture} % Полилиния
        \draw (-1.5, 0) -- (1.5, 0) -- (0, -1.5) -- (0, 1.5); % <-- Путь
        % ^^^ Так как окружение tikzpicture, явное обращение "\tikz" не нужно
        % \draw - основная команда рисования
    \end{tikzpicture}

    % Итого 2 эквивалентных способа обращения к tikz:

    % 1 (для "однострочных" рисунков, такие на практике вряд-ли будут):
    % \tikz какой-то код...;

    % 2 (предпочтительнее для сложных рисунков):
    % \begin{tikzpicture}
    %     какой-то код...;
    % \end{tikzpicture}


    % Круги и эллипсы
    \tikz \draw circle (10pt); % Круг радиусом 10pt (координаты центра - (0, 0) (но там занято => смещает))
    \tikz \draw ellipse (20pt and 10pt); % Эллипс с полуосями 20pt и 10pt (координаты центра - (0, 0) (но там занято => смещает)))
    \tikz \draw [rotate = 30] ellipse (20pt and 10pt); % Эллипс с полуосями 20pt и 10pt, повёрнутый на 30 градусов против часовой стрелки (координаты центра - (0, 0) (но там занято => смещает)))
    \tikz \draw [rotate = -30] ellipse (20pt and 10pt); % Эллипс с полуосями 20pt и 10pt, повёрнутый на 30 градусов по часовой стрелки (координаты центра - (0, 0) (но там занято => смещает)))
    % ^^^ [rotate = 30] - модификатор, его расположение в команде обычно ни на что не влияет,
    % но по правилам хорошего тона их надо располагать до объекта, который они модифицируют (для лучшей читабельности вёрстки)


    % Сетка
    \tikz \draw [xstep = 0.4, ystep = 0.5] (0, 0) grid (2, 2);
    % xstep = 0.4 <-- задаёт расстояние между вертикальными линиями сетки (шаг по оси X) равным 0.4 единицам;
    % ystep = 0.5 <-- задаёт расстояние между горизонтальными линиями сетки (шаг по оси Y) равным 0.5 единицам;
    % (0, 0) <-- координаты левого нижнего угла области (но там занято => смещает)) (можно не указывать);
    % (2, 2) <-- координаты правого верхнего угла области (но там занято => смещает)).

    \tikz \draw [step = 0.5] grid (2, 2);
    % step = 0.5 <-- задаёт расстояние между вертикальными и горизонтальными линиями сетки (шаги по осям X и Y) равным 0.5 единицам;
    % (0, 0) <-- координаты левого нижнего угла области (но там занято => смещает)) (можно не указывать);
    % (2, 2) <-- координаты правого верхнего угла области (но там занято => смещает)).
\end{document}