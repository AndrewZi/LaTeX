\documentclass[10pt, hideallsections]{beamer}

% Этот файл сам по себе не скомпилируется

\usepackage[T2A]{fontenc}
\usepackage[utf8]{inputenc}
\usepackage[english, russian]{babel} % Лучше, когда после fontenс и inputenc

\usepackage{tikz} % Для данного раздела везде % Обратите внимание на соедржание преамбулы!

% В любом случае, LaTeX презентация - это PDF документ!

\begin{document}
    \begin{frame}{Что такое наложения в LaTeX?}
        \begin{enumerate}
            % vvv <1-> (встроенная спецификация, есть у окуружений itemize и enumerate): "-" означает, что с первого нажатия и последнего (т. к. нет второго числа) (здесь: <1-> <=> <1-5>)
            % Место для каждого элемента предусмотрено с самого начала ("раздвиганий" для вставки при необходимости отобразить не будет)
            \item <1-> Одно нажатие\dots % Видно с первого нажатия, т. е. с самого открытия слайда
            \item <4-> Четыре нажатия\dots % Видно с четвёртого нажатия
            \item <2, 4-> Два нажатия, а затем четыре\dots % Видно со второго нажатия, после третьего исчезнет, после четвёртого снова появится
            \item <3-> Три нажатия\dots % Видно с третьего нажатия
            \item <5-> Пять нажатий\dots % Видно с пятого нажатия
        \end{enumerate}
    \end{frame}

    % vvv Другой подход
    % Справка:
    % • pause (не имеет аргументов) - весь текст после этой команды пишется со следующего слайда-наложения;
    % • uncover - текст, который передаётся в качестве аргумента, будет показан на этом и всех последующих слайдах-наложениях;
    % • only - текст, который передаётся в качестве аргумента, будет показан на одном слайде-наложении (осводившееся место может стать занятым).

    % Далее пример аналогичного поведения для текстовых эффектов:
    \begin{frame}{Другой вариант реализации наложений}
        \begin{block}{Красивый пример}
            % Встроенный спецификатор применяется к тексту через слайд-наложение после отображения этого текста
            \textbf <2-> {Вот} \pause
            \textbf <3-> {посмотрите} \pause
            \textbf <4-> {как} \pause
            \textbf <5-> {плавно} \pause
            \textbf <6-> {текст} \pause
            \textbf <7-> {сначала} \pause
            \textbf <8-> {появляется,} \pause
            \textbf <9-> {а} \pause
            \textbf <10-> {потом} \pause
            \textbf <11-> {становится} \pause
            \textbf <12-> {жирным.}
        \end{block}
    \end{frame}
\end{document}