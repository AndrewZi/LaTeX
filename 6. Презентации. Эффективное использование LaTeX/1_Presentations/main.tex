\documentclass[10pt, hideallsubsections]{beamer} % Для создания презентаций
% hideallsubsections <-- навигация только по секциям первого уровня вложенности

\usepackage[T2A]{fontenc}
\usepackage[utf8]{inputenc}
\usepackage[english, russian]{babel} % Лучше, когда после fontenс и inputenc

% Основной принцип - каждый слайд помещается в особое окружение frame

% vvv - рекомендуется размещать в файле преамбулы
\usetheme{CambridgeUS} % Полностью задаёт стиль презентации по заданному шаблону
\usecolortheme{dolphin} % Меняет только (!) цветовую палитру

% Справка. Темы:
% \usefonttheme
% \useinnertheme
% \useoutertheme
% ... См. основные темы beamer LaTeX в Интернете

\setbeamertemplate{navigation symbols}{} % Применение шаблона для beamer
% (здесь: удаление навигационных кнопок)
% ^^^

% Разделы section и subsection остаются такими же, как и в обычных документах.
% Их следует размещать вне окружений frame, чтобы добиться объединения нескольких слайдов
% в одну секцию, что очень полезно для структурированного оглавления презентации.
% В противном случае описанного выше результата не будет, и тогда эти разделы теряют смысл. 

\begin{document}
    \AtBeginSection[]{ % Располагается до всех разделов section, subsection, а также frame
        \begin{frame}{Table of contents} % Для frame "обязателен" (согласно синтаксису LaTeX, но можно и без него) заголок - текст в {...}
            \tableofcontents[currentsection] % currentsection - хранит информацию о текущей секции
        \end{frame}
    }
    \section{Начало презентации}
    \begin{frame}{Слайд 1}
        Начальная информация
    \end{frame}
    \section{Продолжение}
    \begin{frame}{Слайд 2}
        Последующая информация
    \end{frame}
\end{document}