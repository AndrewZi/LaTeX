\documentclass[a4paper, 12pt]{article}

\usepackage[T2A]{fontenc}
\usepackage[utf8]{inputenc}
\usepackage[english, russian]{babel} % Лучше, когда после fontenс и inputenc

\begin{document}
    Обычный текст и вдруг, см. информацию в \cite{b1}.
    % \cite{b1} - ссылка на пукнт из списка литературы с наименованием b1, т. е.
    % на \bibitem{b1} Author 1, Book 1, 2022

    \begin{thebibliography}{99} % 99 - максимальное количество элементов, которые могут быть отображены в списке литературы с использованием нумерации.
% Этот параметр определяет ширину, которую будет занимать номер ссылки в списке.
% Конкретно, число 99 означает, что LaTeX ожидает, что будет не более 99 ссылок в списке литературы, в противном случае ожидается некорректное отображение.
% Если меньше ссылок, чем указанное число, LaTeX просто отобразит номера ссылок, не используя всю выделенную ширину.
        \bibitem{b1} Author 1, Book 1, 2022
        \bibitem{b2} Author 2, Book 2, 2023
        \bibitem{b3} Author 3, Book 3, 2024
    \end{thebibliography}
\end{document}