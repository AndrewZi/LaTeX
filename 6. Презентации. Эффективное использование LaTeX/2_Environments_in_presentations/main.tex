\documentclass[10pt, hideallsubsections]{beamer}

% Этот файл сам по себе не скомпилируется

\usepackage[T2A]{fontenc}
\usepackage[utf8]{inputenc}
\usepackage[english, russian]{babel} % Лучше, когда после fontenс и inputenc

\usepackage{tikz} % Для данного раздела везде % Обратите внимание на соедржание преамбулы!

% Справка. frame. Опции:
% • [b], [c], [t] <-- выравнивание текста (b - внизу слайда, c - в центре слайда, t - вверху слайда);
% • [fragile] <-- используется в случае наличия листингов;
% • [plain] <-- для удаления некоторого оформления.

\begin{document}
    % 1) Пример frame
    \begin{frame}[fragile]{Заголовок} % Заголовка может и не быть,
        % несмотря на то, что это обязательный аргумент, парадокс разрешается тем,
        % что в LaTeX для frame определено несколько конструкций их задания.

        % 2) Пример block
        \begin{block}{\textcolor{red}{Определение}} % \textcolor{red}{Определение} <-- придание цвета элементу, отличного от установленного выбранной темой
        % Заголовка может и не быть,
        % несмотря на то, что это обязательный аргумент, парадокс разрешается тем,
        % что в LaTeX для block определено несколько конструкций их задания.
            Определяем что-то важное\dots
        \end{block}

        % 3) Пример columns (столбцы)
        \begin{columns}
            \begin{column}{40pt} % столбец, 40pt - длина стобца
                Теорема:
            \end{column}
            \begin{column}{40pt} % столбец, 40pt - длина стобца
                Открыл:
            \end{column}
            \begin{column}{3cm} % столбец, 3cm - длина стобца
                Год открытия:
            \end{column}
        \end{columns}
    \end{frame}
\end{document}