\documentclass[a4paper, 12pt]{article}

\usepackage[english, russian]{babel}
\usepackage[T2A]{fontenc}
\usepackage[utf8]{inputenc}

\usepackage{amsthm}
\theoremstyle{plain} % Стиль записи словосочетаний "Теорема N" или "Лемма N"
\newtheorem{theorem}{Теорема} % Создание окружения (внутри окружения своя нумерация)
% \newtheorem* - отсутствие нумерации
\newtheorem{lemma}{Лемма} % Содание окружения (внутри окружения своя нумерация)
% \newtheorem* - отсутствие нумерации

\begin{document}
    \begin{theorem} % Нумерация прописывается автоматически
        \[a^2 + b^2 = c^2\]
    \end{theorem}

    \begin{lemma} % Нумерация прописывается автоматически
        \[0 \cdot a = 0\]
    \end{lemma}

    \begin{theorem} % Нумерация прописывается автоматически
        \[a^2 + b^2 = c^2\]
    \end{theorem}

    \begin{lemma} % Нумерация прописывается автоматически
        \[0 \cdot a = 0\]
    \end{lemma}

    \begin{theorem} % Нумерация прописывается автоматически
        \[a^2 + b^2 = c^2\]
    \end{theorem}

    \begin{lemma} % Нумерация прописывается автоматически
        \[0 \cdot a = 0\]
    \end{lemma}
\end{document}