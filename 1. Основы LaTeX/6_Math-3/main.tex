\documentclass[a4paper, 12pt]{article}

\usepackage[english, russian]{babel}
\usepackage[T2A]{fontenc}
\usepackage[utf8]{inputenc}

\begin{document}
    % C "\" начинается большинство команд в TeX
    Интеграл Пуассона:
    \[\int_{-\infty}^{+\infty} e^{-\frac{x^2}{2}}dx = \sqrt{2\pi}\]
    % Или

    Интеграл Пуассона:
    \[\int_{-\infty}^{+\infty} e^{-\frac{x^2}{2}}dx = \sqrt{2 \pi}\]
    % \int - интеграл
    % -\infty - нижний предел (- бесконечность), +\infty - верхний предел (+ бесконечность)
    % e^{-\frac{x^2}{2}} - подынтегральная функция
    % x - переменная интегрирования
    % \sqrt{2 \pi} - квадратный корень (2) из числа Пи (\pi)

    Не существует в элементарных функциях:
    \[\int_{}^{} e^{-\frac{x^2}{2}}dx = ?\] % Нет пробела после "="!

    Не существует в элементарных функциях:
    \[\int_{}^{} e^{-\frac{x^2}{2}}dx ={ }?\] % Есть пробел после "="

    Бесконечная десятичная дробь:
    \[\sqrt{2}\]
    % Или

    Бесконечная десятичная дробь:
    \[\sqrt2\] % Можно не ставить пробел, если аргумент не символ латиницы
    % Или

    Бесконечная десятичная дробь:
    \[\sqrt 2\] % Наиболее читаемый вариант (работает только с первой цифрой!)
    % Но

    Некоторое число:
    \[\sqrt{a}\]
    % Или

    Некоторое число:
    \[\sqrt a\] % \[\sqrta\] - TeX будет думать, что это команда \sqrta (которой не существует)

    % Также
    \[\sqrt{}\] % Допустимо
    % Но \[\sqrt\] - ошибка компиляции

    Большое число:
    \[\sqrt 1935\] % Корень будет только до символа "1", дальше корень "тянуться" не будет
    % Или

    Большое число:
    \[\sqrt1935\] % Корень будет только до символа "1", дальше корень "тянуться" не будет
    % Но

    Большое число:
    \[\sqrt{1935}\] % Корректное отображение
    % Или

    Большое число:
    \[\sqrt {1935}\] % Корректное отображение

    Также большое число:
    \[\sqrt[]{1935}\] % Лишний код
    % Или

    Также большое число:
    \[\sqrt[3]{1935}\]
    % \[\sqrt{1935}[3]\] - не приведёт к ожидаемому результату!
\end{document}