\documentclass[a4paper, 12pt]{article}

% Этот файл сам по себе не скомпилируется

\usepackage[english, russian]{babel}
\usepackage[T2A]{fontenc}
\usepackage[utf8]{inputenc}

\begin{document}
    % Порядок отображения (можно указывать в коде в другом порядке, но порядок отображения будет следующим):
    \title{Теория возможностей} % Не будет отображено
    \title{Теория принятия решений} % Не будет отображено
    \title{Теория размерностей} % Без \maketitle укзывать нельзя (1)
    \author{Андреев А. А.} % Не будет отображено
    \author{Васильев В. В.} % Не будет отображено
    \author{Иванов И. И.} % Без \maketitle укзывать нельзя (2)
    \date{21.12.2022} % Не будет отображено
    \date{21.12.2023} % Не будет отображено
    \date{21.12.2024} % Будет отображено (3)
    % дата - опционально, если явно не указать, будет отображена текущая дата (12 декабря 2024г.)
    % Чтобы избежать автоматического отображения даты --> \date{}
    \maketitle

    \tableofcontents
    \newpage

    \part{Механика} % Нумерация будет с использованием цифр
        \dots
        \section{Кинематика} % Нумерация будет с использованием цифр
            \dots
            \subsection{Кинематика материальной точки} % Нумерация будет с использованием цифр
                \dots
                \subsubsection{Равномерное прямолинейное движение} % Нумерация будет с использованием цифр
                    \dots
                    \paragraph{Закон в векторном виде} % Нумерация будет с использованием цифр
                        \dots
                        \subparagraph{В проецировании на ось Ox} % Нумерация будет с использованием цифр
                            \dots
        \newpage
        \section[Дин.]{Динамика}
    \newpage
    \appendix % Далее нумерация будет с использованием латинских букв
    \section{Физические величины} % Нумерация будет с использованием латинских букв
        \dots
    \section{Размерности} % Нумерация будет с использованием латинских букв
        \dots
\end{document}