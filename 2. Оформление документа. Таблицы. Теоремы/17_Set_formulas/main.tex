\documentclass[a4paper, 12pt]{article}

% Этот файл сам по себе не скомпилируется

\usepackage[english, russian]{babel}
\usepackage[T2A]{fontenc}
\usepackage[utf8]{inputenc}

\usepackage{amsmath} % Для \limits и \dfrac
\usepackage{amsfonts} % Необходимый пакет

\begin{document}
    \[x \in \mathbb R\] % \mathbb - для написания множеств

    \[\mathbf{Ax} = \mathbf B\] % \mathbf - жирный текст

    \[\varphi \in \mathrm{GL}(V)\] % \mathrm - вертикальный текст

    \[\mathsf E\xi < +\infty\] % \mathsf - шрифт без засечек

    \[\psi \in \mathcal L(V)\]\\ % mathcal - математическая каллиграфия (здесь - преобразование Лапласа)



    Внутри строки: $e^x = \sum_{n = 0}^{+\infty}\frac{x^n}{n!} = \sum\limits_{n = 0}^{+\infty}\dfrac{x^n}{n!}$\\ % Не рекомендуется из-за проблем с отступами между соседними строчками
    % Внутри строки для визуально приятного отображения \limits и \dfrac необходим, на своей строке - необязателен!
    % \limits - позволяет менять отображение на переделы только для операторов, где такое отображение возможно (есть определение оператора с *)

    На своей строке:
    \[e^x = \prod_{n = 0}^{+\infty}\frac{x^n}{n!} = \prod\limits_{n = 0}^{+\infty}\dfrac{x^n}{n!}\]


    \[\varphi \leftrightarrow_e^{\text{Просто текст}} A \in M_n(F)\] % Визуально неприятно

    \[\varphi \xrightarrow[\text{Просто текст}]{e} A \in M_n(F)\] % Визуально приятно
    % e (нижняя надпись) - обязательна; Просто текст (верхняя надпись) - функциональна.


    $\dot a$

    $\ddot a$

    $\mathring a$

    $\hat a$ % "Крышка" фиксированного размера на один символ
    \hfill
    $\widehat{AB}$ % "Крышка" по длине строки в {...}

    $\tilde a$ % "Крышка" фиксированного размера на один символ
    \hfill
    $\widetilde{AB}$ % "Крышка" по длине строки в {...}

    $\bar a$ % "Крышка" фиксированного размера на один символ
    \hfill
    $\overline{AB}$ % "Крышка" по длине строки в {...}

    $\vec a$ % "Крышка" фиксированного размера на один символ
    \hfill
    $\overrightarrow{AB}$ % "Крышка" по длине строки в {...}
\end{document}