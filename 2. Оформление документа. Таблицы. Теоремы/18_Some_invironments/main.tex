\documentclass[a4paper, 12pt]{article}

% Этот файл сам по себе не скомпилируется

\usepackage[english, russian]{babel}
\usepackage[T2A]{fontenc}
\usepackage[utf8]{inputenc}

\usepackage{amsmath} % Для equation*
\usepackage{amsfonts} % Для \mathbb

\begin{document}
    \begin{equation} % Эквивалент \[...\] (формулы нумеруются)
        Ax = b
    \end{equation}

    \begin{equation*} % Эквивалент \[...\] (формулы не нумеруются)
        Ax = b
    \end{equation*}


    \begin{align} % Эквивалент \[...\] (формулы нумеруются)
        Ax = b
        \\ %<-- Перегружен
        Ax^2 + bx + c = 0 % Выравнивание по правому краю
        \\ %<-- Перегружен
        Ax = b
    \end{align}

    \begin{align} % Эквивалент \[...\] (формулы нумеруются)
        Ax &= b % Выравнивание всех формул окружения по "&"
        \\ %<-- Перегружен
        Ax^2 + bx &+ c = 0 % Выравнивание всех формул окружения по "&"
        \\ %<-- Перегружен
        Ax &= b % Выравнивание всех формул окружения по "&"
    \end{align}

    \begin{align*} % Эквивалент \[...\] (формулы не нумеруются)
        Ax = b % Выравнивание по правому краю
        \\ %<-- Перегружен
        Ax^2 + bx + c = 0 % Выравнивание по правому краю
        \\ %<-- Перегружен
        Ax = b % Выравнивание по правому краю
    \end{align*}

    \begin{align*} % Эквивалент \[...\] (формулы не нумеруются)
        Ax &= b % Выравнивание всех формул окружения по "&"
        \\ %<-- Перегружен
        Ax^2 + bx &+ c = 0 % Выравнивание всех формул окружения по "&"
        \\ %<-- Перегружен
        Ax &= b % Выравнивание всех формул окружения по "&"
    \end{align*}


    \begin{gather} % Эквивалент \[...\] (формулы нумеруются)
        Ax = b % Выравнивание по середине
        \\ %<-- Перегружен
        Ax^2 + bx + c = 0 % Выравнивание по середине
        \\ %<-- Перегружен
        Ax = b % Выравнивание по середине
    \end{gather}

    \begin{gather*} % Эквивалент \[...\] (формулы не нумеруются)
        Ax = b % Выравнивание по середине
        \\ %<-- Перегружен
        Ax^2 + bx + c = 0 % Выравнивание по середине
        \\ %<-- Перегружен
        Ax = b % Выравнивание по середине
    \end{gather*}

    
    \begin{multline} % Эквивалент \[...\] (формула нумеруется)
        % Для многострочных формул
        a + b + c + \dots
        \\ %<-- Перегружен
        \dots + e + f + g + \dots
        \\ %<-- Перегружен
        \dots + z = 0
    \end{multline}

    \begin{multline*} % Эквивалент \[...\] (формула не нумеруется)
        % Для многострочных формул
        a + b + c + \dots
        \\ %<-- Перегружен
        \dots + e + f + g + \dots
        \\ %<-- Перегружен
        \dots + z = 0
    \end{multline*}


    % aligned и сases - только внутри каких-либо окружений

    \[f(x) = \begin{cases}
        % Для более приятного восприятия лучше использовать выравнивание по &
        % Это добавит небольшие пробелы между значениями и условиями
        1 & x = 0
        \\ %<-- Перегружен
        0 & x \ne 0 % \ne - символ неравенства
    \end{cases}\]\\

    Функция Дирихле:
    \[D(x) = \begin{cases}
        % Для более приятного восприятия лучше использовать выравнивание по &
        % Это добавит небольшие пробелы между значениями и условиями
        1 & x \in \mathbb Q
        \\ %<-- Перегружен
        0 & x \not\in \mathbb Q % \not\in - символ отсутствия принадлежности
    \end{cases}\]
\end{document}