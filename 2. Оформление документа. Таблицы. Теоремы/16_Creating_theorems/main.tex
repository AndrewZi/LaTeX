\documentclass[a4paper, 12pt]{article}

% Этот файл сам по себе не скомпилируется

\usepackage[english, russian]{babel}
\usepackage[T2A]{fontenc}
\usepackage[utf8]{inputenc}

\usepackage{amsthm} % Пакет для создания теорем

\theoremstyle{plain}
% Встроенные стили теорем:
% plain
% definition
% remark

\newtheorem{theorem}{Теорема}[section] % без * - нумерованные теоремы
% ^^^ Для теоремы: сначала номер секции, потом - номер теоремы в секции (наследование)
\newtheorem{corollary}{Следствие}[theorem]
\newtheorem*{definition}{Определение} % с * - не нумерованные теоремы

\begin{document}
    \section{Теоремы кинематики}

    \begin{theorem}
        Теорема о неразрывности
    \end{theorem}

    \begin{theorem}
        Теорема о полноте
    \end{theorem}


    \section{Теоремы в структуре}

    \begin{theorem}
        Теорема.
    \end{theorem}
    
    \begin{theorem}
        Теорема дополнительная.
    \end{theorem}

    \begin{proof}
        Доказательство тривиально.
        \[x = 0 \qedhere\] % \qedhere - декларирует, в конце какой строки должен быть размещён символ Ч.Т.Д.
% без \qedhere символ Ч.Т.Д. будет размещён на строке после x = 0, что не очень экономит место и вообще визуально неприятно.
    \end{proof}

    \begin{corollary}
        Из теоремы.
    \end{corollary}

    \begin{definition}
        Вектор "--- направленный отрезок.
    \end{definition}
\end{document}