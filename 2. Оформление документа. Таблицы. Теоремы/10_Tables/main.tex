\documentclass[a4paper, 12pt]{article}

% Этот файл сам по себе не скомпилируется

\usepackage[english, russian]{babel}
\usepackage[T2A]{fontenc}
\usepackage[utf8]{inputenc}

\begin{document}
    Рассмотрим следующую таблицу:\\
    \\ % новая строка
    \\ % новая строка
    % vvv - выравание по центру (окружение)
    \centering \begin{tabular}{||c|c|||r|l||} %<-- описание структуры таблицы (таблица - это окружение)
    % | - вертикальная черта в таблице (количество совпадают с количеством в описании структуры)
    % с - выравнивание элементов в столбце по центру
    % r - выранивание элементов в столбце по правому краю
    % l - выравнивание элементов в столбце по левому краю
    % & - разделение элементов по ячейкам в строке
        \hline % верхняя черта таблицы
        1 & x & aligned & aligned \\ %<-- здесь \\ - окончание строки таблицы
        \hline % черта, разделяющая строки таблицы
        % \hline можно расположить несколько черт друг за другом
        2 & y & text to right & text to left \\
        \cline{2-3} % черта, разделяющая строки таблицы в заданном диапазоне (нумерация с 1, здесь со 2 по 3 столбец включительно!)
        % \cline{1-3} можно расположить несколько черт друг за другом
        3 & z & the most biggest text to right & the most biggest text to left \\
        \hline % нижняя черта таблицы
    \end{tabular}
\end{document}